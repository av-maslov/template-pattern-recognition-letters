
% 2023.02.18: This looks most similar to published open access papers

%% 
%% Copyright 2019-2021 Elsevier Ltd
%% 
%% This file is part of the 'CAS Bundle'.
%% --------------------------------------
%% 
%% It may be distributed under the conditions of the LaTeX Project Public
%% License, either version 1.2 of this license or (at your option) any
%% later version.  The latest version of this license is in
%%    http://www.latex-project.org/lppl.txt
%% and version 1.2 or later is part of all distributions of LaTeX
%% version 1999/12/01 or later.
%% 
%% The list of all files belonging to the 'CAS Bundle' is
%% given in the file `manifest.txt'.
%% 
%% Template article for cas-dc documentclass for 
%% double column output.

% Orig: 
\documentclass[a4paper,fleqn]{cas-dc}
% End Orig

% Copied from V2/elsarticle/elsarticle-template-num.tex
%\documentclass[a4paper,12pt]{elsarticle}

% If the frontmatter runs over more than one page
% use the longmktitle option.

%\documentclass[a4paper,fleqn,longmktitle]{cas-dc}

%\usepackage[numbers]{natbib}
%\usepackage[authoryear]{natbib}
\usepackage[authoryear,longnamesfirst]{natbib}
\usepackage{lipsum}
%\usepackage{prletters_copied}
%%%Author macros
%\def\tsc#1{\csdef{#1}{\textsc{\lowercase{#1}}\xspace}}
%\tsc{WGM}
%\tsc{QE}
%%%

% Uncomment and use as if needed
%\newtheorem{theorem}{Theorem}
%\newtheorem{lemma}[theorem]{Lemma}
%\newdefinition{rmk}{Remark}
%\newproof{pf}{Proof}
%\newproof{pot}{Proof of Theorem \ref{thm}}

\begin{document}
	\maketitle
\let\WriteBookmarks\relax
\def\floatpagepagefraction{1}
\def\textpagefraction{.001}

% Short title
\shorttitle{Short title of the paper for running head}

% Short author
\shortauthors{Short author list for running head}

% Main title of the paper
\title [mode = title]{<main title>}  

% Title footnote mark
% eg: \tnotemark[1]
\tnotemark[<tnote number>] 

% Title footnote 1.
% eg: \tnotetext[1]{Title footnote text}
\tnotetext[<tnote number>]{<tnote text>} 

% First author
%
% Options: Use if required
% eg: \author[1,3]{Author Name}[type=editor,
%       style=chinese,
%       auid=000,
%       bioid=1,
%       prefix=Sir,
%       orcid=0000-0000-0000-0000,
%       facebook=<facebook id>,
%       twitter=<twitter id>,
%       linkedin=<linkedin id>,
%       gplus=<gplus id>]

%\author[<aff no>]{<author name>}[<options>]
\author{Author name}

% Corresponding author indication
%\cormark[<corr mark no>]

% Footnote of the first author
%\fnmark[<footnote mark no>]

% Email id of the first author
%\ead{<email address>}

% URL of the first author
%\ead[url]{<URL>}

% Credit authorship
% eg: \credit{Conceptualization of this study, Methodology, Software}
%\credit{<Credit authorship details>}

% Address/affiliation
%\affiliation[<aff no>]{organization={},
%            addressline={}, 
%            city={},
%          citysep={}, % Uncomment if no comma needed between city and postcode
%            postcode={}, 
%            state={},
%            country={}}

%\author[<aff no>]{<author name>}[<options>]
\author{author name}

% Footnote of the second author
\fnmark[2]

% Email id of the second author
\ead{}

% URL of the second author
\ead[url]{}

% Credit authorship
\credit{}

% Address/affiliation
%\affiliation[<aff no>]{organization={},
%            addressline={}, 
%            city={},
%          citysep={}, % Uncomment if no comma needed between city and postcode
%            postcode={}, 
%            state={},
%            country={}}

% Corresponding author text
\cortext[1]{Corresponding author}

% Footnote text
\fntext[1]{}

% For a title note without a number/mark
%\nonumnote{}

% Here goes the abstract
\begin{abstract}
Fusce mauris. Vestibulum luctus nibh at lectus. Sed
bibendum, nulla a faucibus semper, leo velit ultricies tellus,
ac venenatis arcu wisi vel nisl. Vestibulum diam. Aliquam
pellentesque, augue quis sagittis posuere, turpis lacus con-
gue quam, in hendrerit risus eros eget felis. Maecenas eget
erat in sapien mattis porttitor. Vestibulum porttitor. Nulla
facilisi. Sed a turpis eu lacus commodo facilisis. Morbi
fringilla, wisi in dignissim interdum, justo lectus sagittis dui,
et vehicula libero dui cursus dui. Mauris tempor ligula sed
lacus. Duis cursus enim ut augue. Cras ac magna. Cras nulla.
Nulla egestas. Curabitur a leo. Quisque egestas wisi eget
nunc. Nam feugiat lacus vel est. Curabitur consectetuer.
Suspendisse vel felis. Ut lorem lorem, interdum eu, tin-
cidunt sit amet, laoreet vitae, arcu. Aenean faucibus pede eu
ante. Praesent enim elit, rutrum at, molestie non, nonummy
vel, nisl. Ut lectus eros, malesuada sit amet, fermentum eu,
sodales cursus, magna. Donec eu purus. Quisque vehicula,
urna sed ultricies auctor, pede lorem egestas dui, et convallis
elit erat sed nulla. Donec luctus. Curabitur et nunc. Aliquam
dolor odio, commodo pretium, ultricies non, pharetra in,
velit. Integer arcu est, nonummy in, fermentum faucibus,
egestas vel, odio.
\end{abstract}

% Use if graphical abstract is present
%\begin{graphicalabstract}
%\includegraphics{}
%\end{graphicalabstract}

% Research highlights
\begin{highlights}
\item 
\item 
\item 
\end{highlights}

% Keywords
% Each keyword is seperated by \sep
\begin{keywords}
 \sep \sep \sep
\end{keywords}

%\maketitle

% Main text
\section{Section I}\label{sec:one}
\lipsum[1-50]
% Numbered list
% Use the style of numbering in square brackets.
% If nothing is used, default style will be taken.
%\begin{enumerate}[a)]
%\item 
%\item 
%\item 
%\end{enumerate}  

% Unnumbered list
%\begin{itemize}
%\item 
%\item 
%\item 
%\end{itemize}  

% Description list
%\begin{description}
%\item[]
%\item[] 
%\item[] 
%\end{description}  

% Figure
%\begin{figure}[<options>]
%	\centering
%		\includegraphics[<options>]{}
%	  \caption{}\label{fig1}
%\end{figure}


\begin{table}[<options>]
\caption{}\label{tbl1}
\begin{tabular*}{\tblwidth}{@{}LL@{}}
\toprule
  &  \\ % Table header row
\midrule
 & \\
 & \\
 & \\
 & \\
\bottomrule
\end{tabular*}
\end{table}

% Uncomment and use as the case may be
%\begin{theorem} 
%\end{theorem}

% Uncomment and use as the case may be
%\begin{lemma} 
%\end{lemma}

%% The Appendices part is started with the command \appendix;
%% appendix sections are then done as normal sections
%% \appendix

\section{Second}\label{sec:two}

% To print the credit authorship contribution details
\printcredits

%% Loading bibliography style file
%\bibliographystyle{model1-num-names}
\bibliographystyle{cas-model2-names}

% Loading bibliography database
\bibliography{}

% Biography
%\bio{}
% Here goes the biography details.
%\endbio

%\bio{pic1}
% Here goes the biography details.
%\endbio

\end{document}

